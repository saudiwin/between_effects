\include{article_header}

\title{Mid-Dissertation Research Memo}

\begin{document}
	\maketitle
	
\subsection*{Review of Hypotheses}

In my prospectus, I set out an ambitious list of six hypotheses all relating to the political and economic effects of regime transitions on the business community. I will attempt to condense these hypotheses down to those that focus on the actions of businesspeople during the politically unstable period of post-transition. Because of the strong base of cronyism and domestic protectionism underpinning Middle Eastern political economy, it is unlikely that there could be a feasible scenario in which the business community would help create a democratic movement. Rather, the important question in terms of regime transitions is how businesspeople will engage with the counter-revolution that inevitably follows a surprising grass-roots protest movement like the Arab Spring. I want to focus my hypotheses on the decision calculus that propels some businesspeople to become involved in political action while others stay quiescent, and also what are the factors of successful business mobilization.

Hypotheses:

\begin{tabular}{l p{7cm}} 
	H1 & The higher the level of domestic protection of an industry, the more likely businesses in that industry will prefer a reversion to dictatorship after the transition. \\
	H3 & The higher the level of political violence, the more likely that businesspeople will become engaged in political activities because of the fear of further political instability.  \\
	H2 & The longer that firms have been in existence, the less likely that business mobilization will succeed. \\
	H4 & 
\end{tabular}
\end{document}